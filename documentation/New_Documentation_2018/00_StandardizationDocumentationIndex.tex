\documentclass[]{article}
\usepackage{lmodern}
\usepackage{amssymb,amsmath}
\usepackage{ifxetex,ifluatex}
\usepackage{fixltx2e} % provides \textsubscript
\ifnum 0\ifxetex 1\fi\ifluatex 1\fi=0 % if pdftex
  \usepackage[T1]{fontenc}
  \usepackage[utf8]{inputenc}
\else % if luatex or xelatex
  \ifxetex
    \usepackage{mathspec}
  \else
    \usepackage{fontspec}
  \fi
  \defaultfontfeatures{Ligatures=TeX,Scale=MatchLowercase}
\fi
% use upquote if available, for straight quotes in verbatim environments
\IfFileExists{upquote.sty}{\usepackage{upquote}}{}
% use microtype if available
\IfFileExists{microtype.sty}{%
\usepackage{microtype}
\UseMicrotypeSet[protrusion]{basicmath} % disable protrusion for tt fonts
}{}
\usepackage[margin=1in]{geometry}
\usepackage{hyperref}
\hypersetup{unicode=true,
            pdftitle={Standardization \& Balancing Documentation List of documents},
            pdfauthor={Cristina Muschitiello Food and Agriculture Organization of the United Nations},
            pdfborder={0 0 0},
            breaklinks=true}
\urlstyle{same}  % don't use monospace font for urls
\usepackage{graphicx,grffile}
\makeatletter
\def\maxwidth{\ifdim\Gin@nat@width>\linewidth\linewidth\else\Gin@nat@width\fi}
\def\maxheight{\ifdim\Gin@nat@height>\textheight\textheight\else\Gin@nat@height\fi}
\makeatother
% Scale images if necessary, so that they will not overflow the page
% margins by default, and it is still possible to overwrite the defaults
% using explicit options in \includegraphics[width, height, ...]{}
\setkeys{Gin}{width=\maxwidth,height=\maxheight,keepaspectratio}
\IfFileExists{parskip.sty}{%
\usepackage{parskip}
}{% else
\setlength{\parindent}{0pt}
\setlength{\parskip}{6pt plus 2pt minus 1pt}
}
\setlength{\emergencystretch}{3em}  % prevent overfull lines
\providecommand{\tightlist}{%
  \setlength{\itemsep}{0pt}\setlength{\parskip}{0pt}}
\setcounter{secnumdepth}{5}
% Redefines (sub)paragraphs to behave more like sections
\ifx\paragraph\undefined\else
\let\oldparagraph\paragraph
\renewcommand{\paragraph}[1]{\oldparagraph{#1}\mbox{}}
\fi
\ifx\subparagraph\undefined\else
\let\oldsubparagraph\subparagraph
\renewcommand{\subparagraph}[1]{\oldsubparagraph{#1}\mbox{}}
\fi

%%% Use protect on footnotes to avoid problems with footnotes in titles
\let\rmarkdownfootnote\footnote%
\def\footnote{\protect\rmarkdownfootnote}

%%% Change title format to be more compact
\usepackage{titling}

% Create subtitle command for use in maketitle
\newcommand{\subtitle}[1]{
  \posttitle{
    \begin{center}\large#1\end{center}
    }
}

\setlength{\droptitle}{-2em}
  \title{Standardization \& Balancing Documentation\\
List of documents}
  \pretitle{\vspace{\droptitle}\centering\huge}
  \posttitle{\par}
  \author{Cristina Muschitiello\\
Food and Agriculture Organization of the United Nations}
  \preauthor{\centering\large\emph}
  \postauthor{\par}
  \predate{\centering\large\emph}
  \postdate{\par}
  \date{12 June 2018}

\usepackage{lscape}
\usepackage{booktabs}
\usepackage{longtable}
\usepackage{array}
\usepackage{multirow}
\usepackage[table]{xcolor}
\usepackage{wrapfig}
\usepackage{float}
\usepackage{colortbl}
\usepackage{pdflscape}
\usepackage{tabu}
\usepackage{threeparttable}
\usepackage{threeparttablex}
\usepackage[normalem]{ulem}
\usepackage{makecell}

\usepackage{draftwatermark}
\usepackage{makeidx}
\makeindex
\usepackage{float}
\floatplacement{figure}{H}
\usepackage{amsmath}
\usepackage{amssymb}
\usepackage{amsthm}
\usepackage{mathtools}

\begin{document}
\maketitle
\begin{abstract}
A list and brief descritpion of the documents linked to the
Standardization and Balancing in the Food Balance sheet framework is
given.
\end{abstract}

\subsection*{Disclaimer}\label{disclaimer}
\addcontentsline{toc}{subsection}{Disclaimer}

This Working Paper should not be reported as representing the official
view of the FAO. The views expressed in this Working Paper are those of
the author and do not necessarily represent those of the FAO or FAO
policy. Working Papers describe research in progress by the authors and
are published to elicit comments and to further discussion.

This paper is dynamically generated on \today{} and is subject to
changes and updates.

\section*{Introduction}\label{introduction}
\addcontentsline{toc}{section}{Introduction}

The words \emph{Standardization and Balancing} are used for defining the
process of combining commodity balances for creating Food Balance
Sheets. Th FBS framework is defined in a dedicate document that explains
all the steps for creating FBS starting from the initial data
collection. The methodology of \emph{Standardization and Balancing} is,
then explained in a dedicate document. The process is based on a
structured and clear set of relationships between commodities given by
the, so called, \emph{Commodity tree} which is also explained in a
dedicated document. The \emph{Standardization and Balancing} generates
balances for FBS commodities at different levels of aggregation: by FBS
item, by group, by family and Total (by country). There are 3 main
plug-ins connected to this process:

\begin{enumerate}
\def\labelenumi{\arabic{enumi}.}
\tightlist
\item
  \texttt{pullDataToSUA} which created the input dataset for the process
\item
  \texttt{Full\ Standardization\ and\ Balancing}, which performs all the
  steps of the process and saves data in different sessions on 3
  different output datasets
\item
  \texttt{printTree}. This plug-in performs the all process and saves
  the different outputs, plus some detail about commodity tree,
  extraction rates and shares, for a single \emph{country-year-FBS item}
  combination.
\end{enumerate}

All documents describing different aspects of the process are listed
here.

\section{Food Balance Sheet workflow in the Statistical Working
System}\label{food-balance-sheet-workflow-in-the-statistical-working-system}

This documents presents the overall workflow for the production of Food
Balance Sheet inside the SWS. The different SWS's objects involved in
the creation of FBSs are presented and dependencies explained.

\section{Standardization \& Balancing for Food Balance Sheet
Calculation}\label{standardization-balancing-for-food-balance-sheet-calculation}

This document explains all the Standardization and balancing process
from a methodological point of view. All steps are explained and
formalized. No details are contained regarding the R script or the
plug-in in the SWS. Methodology is presented here.

\section{\texorpdfstring{\texttt{faoswsStandardization}:
\texttt{PullDataToSUA}
plug-in}{faoswsStandardization: PullDataToSUA plug-in}}\label{faoswsstandardization-pulldatatosua-plug-in}

Pulling data inside the dataset that is the starting point for
Standardization and Balancing: is the first step of the entire process.
It is first introduced in the Document n.1 while document n.2 presents
what has to be pulled. Here All steps for executing it in the sws are
presented. The plug-in is a module contained in the
\emph{faoswsStandardization} package.

\section{\texorpdfstring{\texttt{faoswsStandardization}:
\texttt{Full\ Standardization\ and\ Balancing}. Data-sets content and
plug-in
execution}{faoswsStandardization: Full Standardization and Balancing. Data-sets content and plug-in execution}}\label{faoswsstandardization-full-standardization-and-balancing.-data-sets-content-and-plug-in-execution}

The plug-in for executing the \emph{Standardization and Balancing} is
presented here in all its steps. The plug-in is a module contained in
the \emph{faoswsStandardization} package.

\section{\texorpdfstring{Standardization \& Balancing:
\texttt{Commodity\ Tree} dataset. Content and usage in the Food Balance
Sheet
framework}{Standardization \& Balancing: Commodity Tree dataset. Content and usage in the Food Balance Sheet framework}}\label{standardization-balancing-commodity-tree-dataset.-content-and-usage-in-the-food-balance-sheet-framework}

Commodity Tree is one of the most important datasets in the process of
producing FBS. It is mentioned in all documents and briefly described.
Here it is fully presented, together with some functions for validating
its content, that are contained in the \emph{faoswsUtil} package.

\section{\texorpdfstring{\texttt{faoswsStandardization}:
\texttt{Print\ tree}
plug-in}{faoswsStandardization: Print tree plug-in}}\label{faoswsstandardization-print-tree-plug-in}

This documents describes how to execute the \texttt{Print\ Tree} plug-in
and its content. This plug-in produces a txt doumnt containing all the
tables related to the Standardization and Balancing, for a single FBS
item.

\section{\texorpdfstring{Standardization \& Balancing:
\texttt{Utilization\ Table} data table. Content and usage in the Food
Balance Sheet
framework}{Standardization \& Balancing: Utilization Table data table. Content and usage in the Food Balance Sheet framework}}\label{standardization-balancing-utilization-table-data-table.-content-and-usage-in-the-food-balance-sheet-framework}


\end{document}
